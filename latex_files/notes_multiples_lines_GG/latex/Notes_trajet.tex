\documentclass[11p]{article}
\usepackage{amsmath}
\usepackage{amsfonts}
\usepackage{bm}

%opening
\title{Transportation network with multiple lines}
\author{notes GG}

\begin{document}

\maketitle

\section{Formalism}

\subsection{The transportation network with multiple lines}

Let $\mathcal{G} = (\mathcal{V}, \mathcal{E})$ be a simple, oriented, and connected graph representing a transportation network between $|\mathcal{V}| = n$ nodes, having $|\mathcal{E}| = m$ edges, and possessing $p$ different transportation lines. Each node belongs to only one line, i.e. $\mathcal{V} = \bigcup_{k=1}^p \mathcal{V}_k$ and $\mathcal{V}_k \cap \mathcal{V}_l = \emptyset, \, \forall k\neq l$, where $\mathcal{V}_k$ represents the set of nodes in line $k$. The edge set $\mathcal{E}$, can also be decomposed with
\begin{equation}
\mathcal{E} = \mathcal{E}_\text{W} \cup \mathcal{E}_\text{B}, \qquad \mathcal{E}_\text{W} := \bigcup_{k=1}^p \mathcal{E}_k, \qquad \mathcal{E}_\text{W} \cap \mathcal{E}_\text{B} = \emptyset, \quad \mathcal{E}_k \cap \mathcal{E}_l = \emptyset, \quad \forall k\neq l.
\end{equation}
where $\mathcal{E}_k$ is the set of edges composing line $k$, $\mathcal{E}_\text{W}$ the set containing all edges inside lines, and $\mathcal{E}_\text{B}$ the set of transfer edges, connecting the different lines. \\
The graph $\mathcal{G}$ can be represented by its adjacency matrix $\mathbf{A} = (a_{ij})$, which can also be decomposed with 
\begin{equation}
\mathbf{A} = \mathbf{A}_\text{W} + \mathbf{A}_\text{B}, \qquad \mathbf{A}_\text{W} = \sum_{k=1}^p \mathbf{A}_k,
\end{equation}
with $\mathbf{A}_k = (a^k_{ij})$ are edges of line $k$, $\mathbf{A}_\text{W} = (a^\text{W}_{ij})$ edges of inside all lines, and $\mathbf{A}_\text{B} = (a^\text{B}_{ij})$ transfer edges. We suppose that there is an uniquely define route inside lines, i.e. 
\begin{equation}
a^k_{i \bullet} \leq 1 \text{ and } a^k_{\bullet i} \leq 1, \qquad \forall i,k.
\end{equation}
where $\bullet$ designates a summation over the replaced index.

\subsection{The origin-destination matrix}

The $(n \times n)$ \emph{origin-destination matrix}, denoted by $\mathbf{N} = (n_{st})$, $n_{st} \geq 0, \; \forall s,t$, contains the flow (e.g. the number of passengers) entering the network in source node $s$ and leaving it in target node $t$. We can denote its margins with 
\begin{align}
\bm{\sigma}_\text{in} := \mathbf{N} \mathbf{e}_n \label{N_rowsum}\\
\bm{\sigma}_\text{out} := \mathbf{N}^\top \mathbf{e}_n \label{N_colsum}
\end{align}
where $\mathbf{e}_n$ is the vector of ones of size $n$. The vector $\bm{\sigma}_\text{in} = (\sigma^\text{in}_i)$ is the \emph{vector of flow entering the network} and $\bm{\sigma}_\text{in} = (\sigma^\text{in}_i)$ is the \emph{vector of flow leaving the network}. We have
\begin{equation}
\sigma^\text{in}_{\bullet} = \sigma^\text{out}_{\bullet}.
\end{equation}

Note that if only $\bm{\sigma}_\text{in}$ and $\bm{\sigma}_\text{out}$ are given, a flow matrix $\mathbf{N}$ can be computed relatively to an \emph{origin-destination affinity matrix} $\mathbf{S} = (s_{st})$, $0 \leq s_{st} \leq 1$, where $s_{st} = 1$ denote a perfect affinity and $s_{st} = 0$ no affinity, through

\begin{equation}
\mathbf{N} = \textbf{Diag}(\mathbf{a}) (\mathbf{S} + \epsilon) \textbf{Diag}(\mathbf{b}),
\end{equation}
where $\textbf{Diag}(.)$ denote the diagonal matrix obtained from a vector, $\epsilon$ a very small quantity, and vectors $\mathbf{a}$ and $\mathbf{b}$ are found through \emph{proportional iterative fitting} algorithm in order to have margin constraints (\ref{N_rowsum}) and (\ref{N_colsum}) respected for $\mathbf{N}$ (a small $\epsilon$ has to be added to $\bm{\sigma}_\text{in}$ and $\bm{\sigma}_\text{out}$ if they possess null components).


\subsection{The flow matrix}

A flow on edges is represented by the $(n \times n)$ \emph{flow matrix} $\mathbf{X} = (x_{ij})$, verifying
\begin{align}
x_{ij} &\geq 0, \qquad \forall i,j, \\
a_{ij} = 0 &\Rightarrow x_{ij} = 0, \qquad \forall i,j, \\ 
x_{i\bullet} + \sigma^\text{out}_i &= x_{\bullet i} + \sigma^\text{in}_i, \qquad \forall i.
\end{align}
Again, we can decompose the flow matrix with 
\begin{equation}
\mathbf{X} = \mathbf{X}_\text{W} + \mathbf{X}_\text{B} \qquad \mathbf{X}_\text{W} := \sum_{k=1}^p \mathbf{X}_k
\end{equation}
where $\mathbf{X}_k$ represent the flow inside line $k$, $\mathbf{X}_\text{W}$ is the flow inside all lines, and $\mathbf{X}_\text{B}$ the flow between lines. This decomposition allows us to define 
the \emph{vector of flow entering lines} $\bm{\rho}_\text{in} = (\rho^\text{in}_i)$ and the \emph{vector of flow leaving lines} $\bm{\rho}_\text{out} = (\rho^\text{out}_i)$, with
\begin{align}
\bm{\rho}_\text{in} &:= \bm{\sigma}_\text{in} + \mathbf{X}^\top_\text{B} \mathbf{e}_n, \\
\bm{\rho}_\text{out} &:= \bm{\sigma}_\text{out} + \mathbf{X}_\text{B} \mathbf{e}_n,
\end{align}
where $\mathbf{e}_n$ is the vector of ones of size $n$. It is easy to see that we still have $\rho^\text{in}_\bullet = \rho^\text{out}_\bullet$. \\

\subsection{Shortest-paths flow}

Let $\mathcal{P}_{st}$ be the set of \emph{admissible} shortest-paths between $s$ and $t$ on $\mathcal{G}$. We can denote by $\text{P}_{st}(i, j)$ the probability of having edge $(i, j) \in \wp$ when drawing a path $\wp$ from $\mathcal{P}_{st}$. We have 
\begin{equation}
\text{P}_{st}(i, j) := \frac{1}{|\mathcal{P}_{st}|}\sum_{\wp \in \mathcal{P}_{st}} \delta((i, j) \in \wp),
\end{equation}
where $\delta(.)$ designate the indicator function. Note that if there is an unique shortest-path between node $s$ and $t$, noted $\wp_{st}$, we have $\text{P}_{st}(i, j) = 1$ if $(i, j) \in \wp_{st}$,  $\text{P}_{st}(i, j) = 0$ otherwise. \\

If we are given an origin-destination matrix $\mathbf{N} = (n_{st})$, we can compute the \emph{shortest-path flow matrix}, noted $\mathbf{X}_\text{sp} = (x^\text{sp}_{ij})$, with
\begin{equation}
x^\text{sp}_{ij} = \sum_{st} \text{P}_{st}(i, j) n_{st} \label{sp_comp}.
\end{equation}
This matrix contains the flow on each edge if we suppose that the flow follows shortest-paths from origin to destination. \\

We can rewrite equation (\ref{sp_comp}) by defining the $(n^2 \times n^2)$ \emph{shortest-path - edge matrix} $\mathbf{P} = (p_{\alpha \beta})$ with
\begin{equation}
p_{\alpha \beta} = \begin{cases}
\text{P}_{st}(i, j) & \text{if } \alpha = s + n(t - 1) \text{ and }  \beta = i + n(j - 1), \\
0 & \text{otherwise}.
\end{cases}
\end{equation}
Then (\ref{sp_comp}) writes
\begin{equation}
\textbf{vec}(\mathbf{X}_\text{sp}) = \mathbf{P}^\top \textbf{vec}(\mathbf{N}),
\end{equation}
where $\textbf{vec}(.)$ denotes the vectorization function of a matrix, obtained by stacking matrix columns on top of one another. \\

From equation (\ref{sp_comp}), we see that 
\begin{equation}
\frac{\partial x^\text{sp}_{ij}}{\partial n_{st}} = \text{P}_{st}(i, j),
\end{equation}
which equal to $1$ if there is a unique shortest-path between $s$ and $t$ and $(i,j)$ belongs to this path. This equation means that if we multiply $n_{st}$ by a factor $\alpha \geq 0$ on each $s, t$ which contains $(i, j)$ on their shortest-paths, the resulting flow on $(i, j)$ will also be multiplied by $\alpha$.

\subsection{Problem definition}

\paragraph{The problem:} We suppose that we know the flow entering and leaving each line, i.e. $\bm{\rho}_\text{in}$ and $\bm{\rho}_\text{out}$, and we want to find origin-destination trajectories $n_{st}$. \\

By setting the problem like that, we easily see that it is ill posed. Several solutions exists, with some of them trival (e.g. units remain on the same line and follow a first-in/first-out scheme), and we need to add some hypotheses to restrain it.

\paragraph{Hypothesis 1:} Trajectories in the network follow shortest-paths from origin $s$ to destination $t$. \\

\paragraph{Hypothesis 2:} The number of trajectories $\mathbf{N} = (n_{st})$ should be as close as possible to $s_{st}$, where $\mathbf{S} = (s_{st})$ is a given affinity matrix between origin and destination nodes, in the sens that
\begin{equation}
	K(\mathbf{N} | \mathbf{S}) := \sum_{st} \frac{n_{st}}{n_{\bullet \bullet}} \log \left( \frac{n_{st}/n_{\bullet \bullet}}{s_{st} / s_{\bullet \bullet}} \right), \label{kl_div}
\end{equation}
i.e. the \emph{Kullback-Leibler divergence} between the probability of selecting an origin-destination path according to $\mathbf{N}$ relatively to the probabilty of selecting an origin-destination path according to $\mathbf{S}$, is minimum. Note that the divergence (\ref{kl_div}) is well defined only if $s_{st} > 0, \; \forall s, t$. \\

With these two additional hypotheses, we can find a solution with the following algorithm.

\subsection{Algorithm}
Set $\bm{\sigma}^{(1)}_\text{in} = \bm{\rho}_\text{in}$, $\bm{\sigma}^{(1)}_\text{out} = \bm{\rho}_\text{out}$, and $\mathbf{S}^{(1)} = \mathbf{S}$. Until convergence, do:
\begin{enumerate}
	\item Compute 
	\begin{equation}
		\mathbf{N}^{(\tau)} = \textbf{Diag}(\mathbf{a}^{(\tau)}) (\mathbf{S}^{(\tau)} + \epsilon)\textbf{Diag}(\mathbf{b}^{(\tau)}),
	\end{equation}
	with proportional iterative fitting, such that $\mathbf{N}^{(\tau)} \mathbf{e}_n = \bm{\sigma}^{(\tau)}_\text{in} + \epsilon$ and $\left(\mathbf{N}^{(\tau)}\right)^\top \mathbf{e}_n = \bm{\sigma}^{(\tau)}_\text{out}  + \epsilon$.
	
	\item Compute the associated shortest-path flow matrix $\mathbf{X}^{(\tau)}$ with
	\begin{equation}
		\textbf{vec}(\mathbf{X}^{(\tau)}) = \mathbf{P}^\top \textbf{vec}(\mathbf{N}^{(\tau)}).
	\end{equation}
	
	\item Compute the vectors of \emph{between-lines flow entering and leaving each nodes}, i.e.
	\begin{align}
		\mathbf{x}^{(\tau)}_\text{B,in} &= (\mathbf{X}^{(\tau)}_\text{B})^\top \mathbf{e}_n \\
		\mathbf{x}^{(\tau)}_\text{B,out} &= \mathbf{X}^{(\tau)}_\text{B} \mathbf{e}_n
	\end{align}

	\item Compute the vectors of \emph{between-line allowed flow entering and leaving each nodes}, written resp. $\tilde{\mathbf{x}}^{(\tau)}_\text{B,in} = (\tilde{x}^{\text{B,in},(\tau)}_i)$ and $\tilde{\mathbf{x}}^{(\tau)}_\text{B,out} = (\tilde{x}^{\text{B,out},(\tau)}_i)$, with 
	\begin{align}
		\tilde{x}^{\text{B,in},(\tau)}_i &= \rho_i^\text{in} \phi \left( \frac{x^{\text{B,in},(\tau)}_i}{\rho_i^\text{in}} \right), \\
		\tilde{x}^{\text{B,out},(\tau)}_i &= \rho_i^\text{out} \phi \left( \frac{x^{\text{B,out},(\tau)}_i}{\rho_i^\text{out}} \right),
	\end{align}
	where $\phi(x)$ is a positive increasing function which should be the identity when $x \to 0$ and with $\phi(x) \leq 1$, $\forall x$. For example
	\begin{enumerate}
		\item $\phi(x) = \min(x, 1)$,
		\item $\phi(x) = \min(x, 1 - \exp(-\lambda x))$, with $\lambda > 0$ a parameter.
	\end{enumerate}

	\item Compute the \emph{between-lines allowed flow on edges}, noted $\widetilde{\mathbf{X}}_\text{B}^{(\tau)} = (\widetilde{x}_{ij}^{\text{B},(\tau)})$ with 
	\begin{equation}
		\widetilde{x}_{ij}^{\text{B},(\tau)} = \begin{cases}
			\min \left(\frac{\tilde{x}^{\text{B,out},(\tau)}_i}{x^{\text{B,out},(\tau)}_i} , \frac{\tilde{x}^{\text{B,in},(\tau)}_j}{x^{\text{B,in},(\tau)}_j} \right)  x^{\text{B},(\tau)}_{ij} & \text{if } x^{\text{B},(\tau)}_{ij} > 0\\
			0 & \text{otherwise}
			\end{cases}
	\end{equation}
	\item Update the flow entering and leaving the network with
		\begin{align}
			\bm{\sigma}^{(\tau+1)}_\text{in} &= \bm{\rho}_\text{in} -  (\widetilde{\mathbf{X}}^{(\tau)}_\text{B})^\top \mathbf{e}_n, \\
			\bm{\sigma}^{(\tau+1)}_\text{out} &= \bm{\rho}_\text{out} -  \widetilde{\mathbf{X}}^{(\tau)}_\text{B} \mathbf{e}_n.
		\end{align}
	\item Compute the \emph{reducing factor matrix} $\mathbf{R}^{(\tau)} = (r^{(\tau)}_{st})$ with 
	\begin{equation}
		r^{(\tau)}_{st} = 1 - \max_{ij} \left( \text{P}_{st}(i, j) \frac{x_{ij}^{\text{B},(\tau)}  - \widetilde{x}_{ij}^{\text{B},(\tau)}}{x_{ij}^{\text{B},(\tau)} + \epsilon} \right).
	\end{equation}
	\item Update the origin-destination affinity matrix with
	\begin{equation}
		\mathbf{S}^{(\tau + 1)} = \mathbf{S}^{(\tau)} \odot \mathbf{R}^{(\tau)},
	\end{equation}
	where $\odot$ designates the Hadamard (component-wise) product of matrices. 
\end{enumerate}


\end{document}
