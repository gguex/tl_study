%% BioMed_Central_Tex_Template_v1.06
%%                                      %
%  bmc_article.tex            ver: 1.06 %
%                                       %

%%IMPORTANT: do not delete the first line of this template
%%It must be present to enable the BMC Submission system to
%%recognise this template!!

%%% loading packages, author definitions

%\documentclass[twocolumn]{bmcart}% uncomment this for two column layout and comment line below
\documentclass{bmcart}

\usepackage{mathptmx}       % selects Times Roman as basic font
\usepackage{helvet}         % selects Helvetica as sans-serif font
\usepackage{courier}        % selects Courier as typewriter font
\usepackage{type1cm}        % activate if the above 3 fonts are not available on your system
\usepackage{makeidx}         % allows index generation
\usepackage{graphicx}        % standard LaTeX graphics tool when including figure files
\usepackage{multicol}        % used for the two-column index
\usepackage[bottom]{footmisc} % places footnotes at page bottom
\usepackage{subfig}
\usepackage{amsfonts}
\usepackage[cmex10]{amsmath}
\usepackage{float}
\usepackage[utf8]{inputenc}

\usepackage{bm} 
\usepackage{amsmath}
\usepackage{amssymb}

\usepackage{caption}

%%%%%%%%%%%%%%%%%%%%%%%%%%%%%%%%%%%%%%%%%%%%%%%%%
%%                                             %%
%%  If you wish to display your graphics for   %%
%%  your own use using includegraphic or       %%
%%  includegraphics, then comment out the      %%
%%  following two lines of code.               %%
%%  NB: These line *must* be included when     %%
%%  submitting to BMC.                         %%
%%  All figure files must be submitted as      %%
%%  separate graphics through the BMC          %%
%%  submission process, not included in the    %%
%%  submitted article.                         %%
%%                                             %%
%%%%%%%%%%%%%%%%%%%%%%%%%%%%%%%%%%%%%%%%%%%%%%%%%

\def\includegraphic{}
\def\includegraphics{}

%%% Put your definitions there:
\startlocaldefs
\endlocaldefs

%%% Begin ...
\begin{document}

%%% Start of article front matter
\begin{frontmatter}

\begin{fmbox}
\dochead{Research}

% Document information

\title{Estimation of flow trajectories in a multi-lines network}

\author[
  addressref={aff1},                   
  corref={aff1},                       
  email={gguex@unil.ch}
]{\inits{G.G.}\fnm{Guillaume} \snm{Guex}}
\author[
  addressref={aff2},
  email={rloup@unil.ch}
]{\inits{R.L.}\fnm{Romain} \snm{Loup}}
\author[
addressref={aff1, aff2},
email={fbavaud@unil.ch}
]{\inits{F.B.}\fnm{François} \snm{Bavaud}}

\address[id=aff1]{%                           
  \orgdiv{Departement of Language and Information Sciences},             
  \orgname{University of Lausanne},          
  \city{Lausanne},                              
  \cny{Switzerland}                                   
}
\address[id=aff2]{%
  \orgdiv{Institute of Geography and Sustainability},
  \orgname{University of Lausanne},          
  \city{Lausanne},                       
  \cny{Switzerland} 
}

\end{fmbox}

% The Abstract begins here                  

\begin{abstractbox}

\begin{abstract} % abstract
Blabla
\end{abstract}

% Keywords begin here  

\begin{keyword}
\kwd{sample}
\kwd{article}
\kwd{author}
\end{keyword}

% MSC classifications codes, if any
%\begin{keyword}[class=AMS]
%\kwd[Primary ]{}
%\kwd{}
%\kwd[; secondary ]{}
%\end{keyword}

\end{abstractbox}

\end{frontmatter}

% Main Body

%% --------------------------------- INTRODUCTION

\section{Introduction}

\subsection{Context}

\subsection{Statement of the problem}

\subsection{Related Works}

%% --------------------------------- FORMALISM

\section{Formalism}

\subsection{Notations}

\subsection{Problem}

\subsection{Solution}

%% --------------------------------- CASE STUDIES

\section{Case Studies}

\subsection{Toy Examples}

\subsection{Real Data}

%% --------------------------------- CONCLUSION

\section{Conclusion}


\section*{Appendix}
Text for this section\ldots

%%%%%%%%%%%%%%%%%%%%%%%%%%%%%%%%%%%%%%%%%%%%%%
%%                                          %%
%% Backmatter begins here                   %%
%%                                          %%
%%%%%%%%%%%%%%%%%%%%%%%%%%%%%%%%%%%%%%%%%%%%%%

\begin{backmatter}

\section*{Acknowledgements}%% if any
Text for this section\ldots

\section*{Funding}%% if any
Text for this section\ldots

\section*{Abbreviations}%% if any
Text for this section\ldots

\section*{Availability of data and materials}%% if any
Text for this section\ldots

\section*{Ethics approval and consent to participate}%% if any
Text for this section\ldots

\section*{Competing interests}
The authors declare that they have no competing interests.

\section*{Consent for publication}%% if any
Text for this section\ldots

\section*{Authors' contributions}
Text for this section \ldots

\section*{Authors' information}%% if any
Text for this section\ldots

%%%%%%%%%%%%%%%%%%%%%%%%%%%%%%%%%%%%%%%%%%%%%%%%%%%%%%%%%%%%%
%%                  The Bibliography                       %%
%%                                                         %%
%%  Bmc_mathpys.bst  will be used to                       %%
%%  create a .BBL file for submission.                     %%
%%  After submission of the .TEX file,                     %%
%%  you will be prompted to submit your .BBL file.         %%
%%                                                         %%
%%                                                         %%
%%  Note that the displayed Bibliography will not          %%
%%  necessarily be rendered by Latex exactly as specified  %%
%%  in the online Instructions for Authors.                %%
%%                                                         %%
%%%%%%%%%%%%%%%%%%%%%%%%%%%%%%%%%%%%%%%%%%%%%%%%%%%%%%%%%%%%%

% if your bibliography is in bibtex format, use those commands:
\bibliographystyle{bmc-mathphys} % Style BST file (bmc-mathphys, vancouver, spbasic).
\bibliography{bmc_article}      % Bibliography file (usually '*.bib' )
% for author-year bibliography (bmc-mathphys or spbasic)
% a) write to bib file (bmc-mathphys only)
% @settings{label, options="nameyear"}
% b) uncomment next line
%\nocite{label}

% or include bibliography directly:
% \begin{thebibliography}
% \bibitem{b1}
% \end{thebibliography}

%%%%%%%%%%%%%%%%%%%%%%%%%%%%%%%%%%%
%%                               %%
%% Figures                       %%
%%                               %%
%% NB: this is for captions and  %%
%% Titles. All graphics must be  %%
%% submitted separately and NOT  %%
%% included in the Tex document  %%
%%                               %%
%%%%%%%%%%%%%%%%%%%%%%%%%%%%%%%%%%%

%%
%% Do not use \listoffigures as most will included as separate files

\section*{Figures}
  \begin{figure}[h!]
  \caption{Sample figure title}
\end{figure}

\begin{figure}[h!]
  \caption{Sample figure title}
\end{figure}

%%%%%%%%%%%%%%%%%%%%%%%%%%%%%%%%%%%
%%                               %%
%% Tables                        %%
%%                               %%
%%%%%%%%%%%%%%%%%%%%%%%%%%%%%%%%%%%

%% Use of \listoftables is discouraged.
%%
\section*{Tables}
\begin{table}[h!]
\caption{Sample table title. This is where the description of the table should go}
  \begin{tabular}{cccc}
    \hline
    & B1  &B2   & B3\\ \hline
    A1 & 0.1 & 0.2 & 0.3\\
    A2 & ... & ..  & .\\
    A3 & ..  & .   & .\\ \hline
  \end{tabular}
\end{table}

%%%%%%%%%%%%%%%%%%%%%%%%%%%%%%%%%%%
%%                               %%
%% Additional Files              %%
%%                               %%
%%%%%%%%%%%%%%%%%%%%%%%%%%%%%%%%%%%

\section*{Additional Files}
  \subsection*{Additional file 1 --- Sample additional file title}
    Additional file descriptions text (including details of how to
    view the file, if it is in a non-standard format or the file extension).  This might
    refer to a multi-page table or a figure.

  \subsection*{Additional file 2 --- Sample additional file title}
    Additional file descriptions text.

\end{backmatter}
\end{document}
