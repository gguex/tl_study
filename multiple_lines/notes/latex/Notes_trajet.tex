\documentclass[11p]{article}
\usepackage{amsmath}
\usepackage{amsfonts}
\usepackage{bm}

%opening
\title{Bus lines, multilines approach}
\author{notes GG}

\begin{document}

\maketitle

\section{Formalism}

\subsection{Network definition}

Let $\mathcal{G} = (\mathcal{V}, \mathcal{E})$ be a oriented graph and $\mathbf{A}$ its adjacency matrix, representing a multimodal transportation network between $|\mathcal{V}| = n$ stops and possessing $l$ lines. Each stop belong to only one line, i.e. $\mathcal{V} = \bigcup_{k=1}^l \mathcal{V}_k$ and $\bigcap_{k=1}^l \mathcal{V}_k = \emptyset$, where $\mathcal{V}_k$ represents the set of nodes in line $k$. The edge set $\mathcal{E}$, can also be decomposed with $\mathcal{E} = \left(\bigcup_{k=1}^l \mathcal{E}_k\right) \cup \mathcal{E}_\text{trsf}$, where $\mathcal{E}_k$ contains edges connecting node inside transportation line $k$, and $\mathcal{E}_\text{trsf}$ contains edges permitting transfer between different transportation lines. We suppose that edges belong to only a unique set, i.e. $\mathbf{A} = \sum_{k=1}^l \mathbf{A}_k + \mathbf{A}_\text{trsf}$, and there are an uniquely define route inside lines, i.e. $a^k_{i \bullet} \leq 1$ and $a^k_{\bullet i} \leq 1$ $\forall i,k$. Transfer edges also define a subset of nodes $\mathcal{V}_\text{f} \subset \mathcal{V}$, containing \emph{free nodes}, i.e. nodes connected to a transfer edge. We have $i \in \mathcal{V}_\text{f} \iff a^\text{trsf}_{i \bullet} + a^\text{trsf}_{\bullet i} > 0$.

\subsection{Flow definition}


A matrix $\mathbf{N} = (n_{ij})$, verifying:
\begin{enumerate}
	\item $n_{ij} \geq 0$
	\item $a_{ij} = 0 \Rightarrow n_{ij} = 0$
\end{enumerate}
is a \emph{flow matrix} defined on $\mathcal{G}$. Again, we can decompose this matrix with 
\begin{equation}
\mathbf{N} = \mathbf{N}_\text{W} + \mathbf{N}_\text{B} \qquad \mathbf{N}_\text{W} := \sum_{k=1}^l \mathbf{N}_k
\end{equation}
where the flow $\mathbf{N}_\text{W}$ is the flow inside the lines, $\mathbf{N}_\text{B}$ the flow between lines, and $\mathbf{N}_l$ the flow inside line $l$.

Suppose that we possess two $n$-length vectors $\mathbf{l}_\text{in} = (l^\text{in}_i)$ and $\mathbf{l}_\text{out} = (l^\text{out}_i)$, representing ,respectively, the flow entering the line associated with stop $i$ and going out of this line at stop $i$. These vectors can in fact be expressed with: 
\begin{align}
\mathbf{l}_\text{in} &= \bm{\sigma}_\text{in} + 
\end{align}
\end{document}
